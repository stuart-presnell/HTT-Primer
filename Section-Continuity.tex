\section{Continuity}

In this section we introduce a central idea of Homotopy Type Theory that, until now, we have been able to get away with ignoring.  This is the idea of \terminology{continuity}.  Continuity is the starting point whereby homotopy enters Homotopy Type Theory!  It will be an important part of motivating the definition of \emph{identity types} in Section \ref{}, and is central to the correct interpretation of the quantifiers introduced in Section \ref{} (as we'll see later).  We therefore need to take care to introduce, understand, and motivate these ideas properly.

Initially it might seem most odd that an idea like continuity should play any role at all in a putative foundation for mathematics.
Continuity enters into mathematics via \emph{topology}, whose definitions are initially motivated by considerations of \emph{continua} in the form of rich mathematical entities like the Euclidean plane and Riemannian manifolds.
But the definition of these structures requires mathematical technology like fields, and in particular the field of real numbers $\RR$, which clearly do not belong in the foundations.

We subsequently abstract away from these examples to give a definition of topological space that depends on nothing more sophisticated than the language of set theory -- subsets, unions, intersections, etc.  While this captures the formalisation of the idea, it leaves behind the intuition that motivates the subject.  Put another way: if we were asked to describe our pre-mathematical ideas of what ``continuous'' means we would not immediately reach for a \emph{discrete} collection of objects!
So while we could define point set topology using just the primitives of a foundational theory, without the richer examples to motivate these ideas we have no coherent justification for doing so.

How, then, can a concept that (usually) naturally occurs rather late in the development of mathematical ideas be a central idea of a supposedly \emph{foundational} theory?
To answer this we need to find a conceptually simpler approach -- a chain of ideas that leads to continuity without going through rich structures based on continua.  We will develop this alternative approach in the present section.
First we'll define continuity as it appears in classical mathematics: as a property of maps between topological spaces, with the particular example of continuous functions $\RR \to \RR$ being the easiest to understand.  We will then see how a version of the same ideas arises in the context of \emph{computation}.  (We might also see how the latter idea is formalised in the definition of \emph{Scott continuity}.)



\newpage
\subsection{Topology}

A \terminology{topological space} is a pair consisting of a set $X$ and a collection $T$ of subsets of $X$ (i.e. $T \subseteq 2^T$), which we call the \emph{open} subsets of $X$, satisfying the following:
\begin{enumerate}[(i)]
\item $\varnothing$ and $X$ are both open;
\item the union of any collection of open subsets of $X$ is open;
\item the intersection of any finite collection of open subsets of $X$ is open.
\end{enumerate}

For example the ``open intervals'' $(a,b)$ on the real line $\RR$, defined as
\[
(a,b) \DefinedAs \setSuchThat{r \in \RR}{a < r \;\AND\; r < b}
\]
are open subsets of $\RR$, and any open subset of $\RR$ is the union of open intervals.  Thus, for example, for any $r \in \RR$ and any positive $e \in \RR$, the interval 
\[
(r-e, r+e) \DefinedAs \setSuchThat{r' \in \RR}{|r' - r| < e}
\] 
is an open interval of $\RR$.  We write $B_e(r)$ for this interval.

A \terminology{continuous function} is a function between topological spaces
\[
f: (X,T) \to (X', T')
\] 
such that for every open set $V \in T'$, the inverse image under $f$ is open in $(X,T)$:
\[
V \in T' 
\;\IMPLIES\;
f^{-1}(V) \DefinedAs \setSuchThat{x \in X}{f(x) \in V} \in T
\]


In the special case of functions $\RR \to \RR$, where $\RR$ is equipped with the open interval topology described above, we can state the definition a little differently.  We say that $f$ is continuous at $x_0 \in \RR$ iff 
\[
\forall \epsilon >0,\,
\exists \delta > 0\,
\mbox{ such that }
\forall x \in \RR,\,
[
|x - x_0| < \delta
\IMPLIES
|f(x) - f(x_0)| < \epsilon
]
\]
or, phrasing it in terms of the open intervals:
\[
\forall \epsilon >0,\,
\exists \delta > 0\,
\mbox{ such that }
\forall x \in B_{\delta}(x_0),\, 
f(x) \in B_{\epsilon}(f(x_0))
\]
This captures the idea that small changes in $x$ (i.e. changes smaller than $\delta$) only lead to small changes in $f(x)$ (i.e. changes smaller than $\epsilon$).
If we want to ensure that we hit an output value less than some $\epsilon$ from $f(x_0)$ then we can always do so by using an input value within $\delta$ of $x_0$, and for every finite $\epsilon$ there is a finite $\delta$ that guarantees this.


\newpage
\subsection{Streams}

In Section \ref{} we saw the recursive scheme that lets us define functions $\List{\type{A}}$ to some other type (where the terms of $\List{\type{A}}$, recall, are finite sequences of terms of \type{A}). In particular, this allowed us to define the natural numbers $\NN$ as $\List{\type{1}}$, where every term of $\NN$ is a finite sequence of $\ast$s which is therefore a natural number.

Since there are an infinite number of terms of $\NN$, any function  $\term{f}:\NN \to \type{A}$ corresponds to an \emph{infinite} collection of terms of \type{A}, namely $\term{f}(\zN)$, $\term{f}(1)$, $\term{f}(2)$, \ldots  What can we do with such a thing?

In practice, of course, we can only ever access a finite number of these terms, since we can only ever do a finite amount of computation.  Thus in practice $\term{f}$ is not really an infinite collection of values, but rather a finite but unbounded -- or what is sometimes called a \emph{potentially infinite} -- collection of terms of \type{A}.  Such potentially infinite collections are called \terminology{streams}.

Note also that, since any function $\term{f}:\NN \to \type{A}$ must be defined recursively (see Section \ref{}), in the course of producing a particular $\term{f}(\term{n})$ we must evaluate \term{f} at every number smaller than \term{n}.  Thus whenever we obtain a value of the stream we must produce an initial segment of it.  
We could therefore think of the stream instead as a function $\term{g}: \NN \to \List{\type{A}}$ that returns the first \term{n} elements of the stream $\term{f}$ when given input \term{n}.

Now consider a function from streams to streams: 
\[
\term{h}: (\NN \to \type{A}) \to (\NN \to \type{B})
\]
Given a stream $\term{f}:\NN \to \type{A}$, the output is a stream $\term{h}(\term{f}): \NN \to \type{B}$.  When we give an input \term{n} to $\term{h}(\term{f})$ it must produce some term of \type{B}, which will of course in general depend upon \term{n} and the original stream \term{f}.
Since we require that all functions terminate on all inputs (i.e. that all functions are \emph{total}), the computation of $\term{h}(\term{f})(\term{n})$ cannot depend upon the \emph{entire} infinite sequence of values that could in principle be produced by \term{f}.  Rather, it can can only depend upon a \emph{finite} initial segment of $\term{f}$.  This must hold for any \term{n} that we give to $\term{h}(\term{f})$.  We don't require that \emph{the same} initial segment of \term{f} be sufficient to work out all values of $\term{h}(\term{f})$ for all inputs \term{n}.  But for every \term{n} there must be \emph{some} initial segment of \term{f} that is sufficient to calculate $\term{h}(\term{f})(\term{n})$.  That is:
\[
\forall \term{n}:\NN,\,
\exists \term{m}:\NN, \,
\mbox{ such that } 
[\term{f}(\zN), \term{f}(1), \ldots, \term{f}(\term{m})]
\mbox{ is sufficient to calculate }
\term{h}(\term{f})(\term{n})
\]
This requirement on \term{h} is very similar in form to the definition of continuity for functions $\RR \to \RR$ given above!  With some more work we can demonstrate that this is indeed a form of continuity requirement on \term{h}.  Thus we can start to see hints of how continuity might enter our foundational story.


\subsection{Topology for Streams}

Consider the type of streams, i.e. functions from $\NN$ to any type.  Temporarily, let's think of this type as just a set of terms.  Can we put a topology on this set such that the above requirement on the function $\term{h}$ is just continuity with respect to this topology?
